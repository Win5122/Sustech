\documentclass[lettersize,journal]{IEEEtran}
\usepackage{blindtext}
\usepackage{tikz}
\usepackage{algorithm}
\usepackage{algpseudocode}
\usepackage{chronology}
\usepackage{graphicx}
\usepackage{subcaption}
\usepackage{amssymb}
\usepackage{amsmath}
\usepackage{array}
\usepackage{tabularx}      % 更灵活的表格宽度控制
\usepackage{ragged2e}      % 更强大的对齐控制
\usepackage{booktabs}      % 更专业的表格线(可选)
\renewcommand{\arraystretch}{1.3} % 调整行高
\usetikzlibrary{graphs, positioning, quotes, shapes.geometric}

\title{Beef Aging Technology: A Gastronomic Revolution in Industrialization}
\author{Wang Qianyi      12111003}

\begin{document}
  \maketitle

\section{Introduction}

Beef aging technology, a seemingly simple food processing method, is, in essence, a product of the interplay between industrialization and urban civilization. From the breakthroughs in refrigeration technology in the late 19th century to the popularization of vacuum packaging in the twentieth century, aging technology not only reshaped the spatio-temporal boundaries of meat consumption, but also profoundly reflected modern humanity's eternal struggle between efficiency and flavor.

\section{Historical context}

The Taste Dilemma of Urbanization

\subsection{The 19th Century Urban Crisis}

\subsubsection{}

Postindustrial revolution, European cities experienced explosive population growth (e.g., London's population surged from 1 million to 6.5 million between 1800 and 1900), collapsing traditional meat supply chains. Fresh beef spoiled rapidly, failing to meet urban demand.

\subsubsection{Technology-Driven Progress}

In 1842, German engineer Carl von Linde invented the first compression refrigeration machine, ushering in the era of cold chain technology. New York meat traders quickly adopted icehouses to store beef, reducing loss rates from 40\% to 15\%.

\subsection{The Culinary Awakening of the Middle Class}

\subsubsection{}

In the late 19th century, American "beef barons" like Gustavus Swift utilized railway cold-chain networks to transport Midwestern beef to eastern cities. Dry-aging techniques (hanging beef in a controlled temperature and humidity environment) became a staple in high-end restaurants, symbolizing the wealth and taste of the emerging bourgeoisie.

\section{Development process}

From Empiricism to Science

\subsection{Industrialization of Dry-Aging (1880–1960)}

\subsubsection{Standardized Processes}

Chicago meatpacking firms published the Aging Process Manual, specifying parameters such as temperature (1 to 4 °C), humidity (80 to 85\%), and airflow (0.5 m/s).

\subsubsection{Equipment Innovation}

In the 1920s, ventilation control systems replaced natural ventilation. The New York Meat Exchange installed the first mechanical dehumidification device, shortening the aging cycle from 28 to 21 days.

\subsection{The Rise of Wet-Aging (1950–1990)}

\subsubsection{Vacuum Packaging Revolution}

In 1956, Cryovac introduced commercial vacuum packaging machines, enabling wet-aged beef to enter supermarkets.

\subsubsection{Technological Advantages}

Eliminating the need to trim the dry, hard outer layers, the loss rates decreased from 30\% (dry aging) to 5\%, reducing costs by 40\%. Walmart's large-scale procurement in the 1980s further propelled its mainstream adoption.

\section{Impacts}

Reshaping Modern Urban Life

\begin{table}[H]
    \centering
    \begin{tabularx}{0.5\textwidth}{@{}>{\RaggedRight}p{1.5cm}|>{\RaggedRight}p{2.8cm}|>{\RaggedRight\arraybackslash}X@{}} % 调整列宽比例
        \hline
        \textbf{Dimension} & \textbf{Impact} & \textbf{Case / Data} \\
        \hline
        Social & Democratization of Dining: Aged beef transitioned from five-star restaurants to middle-class homes & U.S. supermarket sales of wet-aged beef increased by 800\% between 1950 and 1975 \\
        \hline
        Economic & Industrial Chain Reconfiguration: Spurring growth in cold storage construction, plastic packaging, and related industries & Global cold chain market expanded 12-fold from 1960 to 1990, with cold storage capacity growing at an average annual rate of 7.3\% (FAO data) \\
        \hline
        Cultural & Transformation of Taste Aesthetics: Consumers accepted industrial standards prioritizing tenderness over flavor & A 1985 survey by the American Restaurant Association found 68\% of consumers chose wet-aged beef for its "more consistent texture" \\
        \hline
        Ecological & The Resource Paradox: Vacuum packaging reduced waste but increased plastic waste & Annual incremental meat packaging waste in the U.S. reached 200,000 tons in the 1990s (EPA report) \\
        \hline
        Spatial & Industrial Geographic Reshaping: Giant cold storage clusters emerged in urban fringe areas & Chicago's meatpacking district saw a fourfold increase in cold storage area between 1950 and 1980, forming a "frozen zone" industrial landscape \\
        \hline
        Philosophical & Authenticity Debate: Standardized production vs. traditional craftsmanship & French gastronome Brillat-Savarin protested in 1975: "Vacuum bags are murdering the soul of beef" \\
        \hline
    \end{tabularx}
    \caption{Multi-dimensional impacts of aged beef industrialization}
    \label{tab:impacts}
\end{table}

\section{Conclusion}

The Modern Dilemma of Efficiency vs. Flavor

The evolution of beef aging technology essentially represents industrialization's "spatiotemporal compression" of the food chain—cold chains extending meat's lifecycle, vacuum packaging eliminating geographical constraints, yet standardized production blurring individual differences. When choosing a packet of wet-aged steak from a supermarket freezer today, perhaps we should ponder: Has technological progress brought us gustatory liberation, or homogenization of taste?

\section{References}

\subsection{Technical History}

1. Refrigeration Technology Development in the Meat Industry (1840–1960). Archives of the American Meat Institute, 1962.

2. Cryovac Inc. Patent Document. "Application of Vacuum Packaging in Meat Preservation". 1956.

\subsection{Social Impact}

3. Food and Agriculture Organization of the United Nations. The Cold Chain Revolution: Transformation of the 20th Century Food System. 1995.

4. "Beef in the Plastic Age: The Price of Convenience." The New York Times, 1975.

\subsection{Cultural Critique}

5. Brillat-Savarin. The Physiology of Taste. Paris: Institut de France, 1975.

6. Baumman, Zygmunt. Liquid Modernity. Cambridge: Polity Press, 1998.

\end{document}
